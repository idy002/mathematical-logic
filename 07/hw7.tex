\documentclass[10pt,a4paper]{article}
\usepackage[utf8]{inputenc}
\usepackage{amsthm}
\usepackage{enumitem} 
\usepackage{amsmath}
\usepackage{amsfonts}
\usepackage{amssymb}
\usepackage{graphicx}
\author{Ding Yaoyao}
\title{Mathematical Logic Homework 7}

\newenvironment{sol}[1]
{\par\vspace{3mm}\noindent{\it Solution #1}.}
{\qed}

\newcommand{\abs}[1]{\lvert#1\rvert}
\newcommand{\fA}{\mathfrak{A}}
\newcommand{\fB}{\mathfrak{B}}
\newcommand{\fC}{\mathfrak{C}}
\newcommand{\fI}{\mathfrak{I}}
\newcommand{\fJ}{\mathfrak{J}}
\newcommand{\fT}{\mathfrak{T}}
\newcommand{\cA}{\mathcal{A}}
\newcommand{\cB}{\mathcal{B}}
\begin{document}
	\maketitle
	
	\begin{sol}{7.1}
		By Completeness Theorem, we can derive $\Theta \models \varphi$ from $\Theta \vdash \varphi$. Then for any $S$-interpretation $\mathfrak{I}$ 
		$$
			\fI \models \Theta \text{ implies } \fI \models \varphi
		$$
		We can construct a $S_0$-interpretation $\fI'$ by retaining the symbols occurring in $\Theta$ and $\varphi$ and keep their interpretation unchanged. By Coincidence Lemma, 
		$$
			\fI' \models \Theta \text{ implies } \fI' \models \varphi 
		$$
		It's obvious that any $S_0$-interpretation can be expanded to a $S$-interpretation without changing the interpretation of symbols in $S_0$. So for any $S_0$-interpretation $\fI'$,
		$$
					\fI' \models \Theta \text{ implies } \fI' \models \varphi 
		$$
		which means $\Theta \models \varphi$ (Now $\varphi$ is a $S_0$-formula and so does the formulas in $\Theta$ ). By Completeness Theorem, $\Theta \vdash \varphi$ and every formula occurs in the proof is a $S_0$-formula. 
		
	\end{sol}

	\begin{sol}{7.2}
	\end{sol}

\end{document}