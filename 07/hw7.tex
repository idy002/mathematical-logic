\documentclass[10pt,a4paper]{article}
\usepackage[utf8]{inputenc}
\usepackage{amsthm}
\usepackage{enumitem} 
\usepackage{amsmath}
\usepackage{amsfonts}
\usepackage{amssymb}
\usepackage{graphicx}

\author{Ding Yaoyao}
\title{Mathematical Logic Homework 7}

\newenvironment{sol}[1]
{\par\vspace{3mm}\noindent{\it Solution #1}.}
{\qed}

\newenvironment{thm}[1]
{\par\vspace{3mm}\noindent{\textbf{#1}}.\quad}{\\\\}

\newcommand{\abs}[1]{\lvert#1\rvert}
\newcommand{\fA}{\mathfrak{A}}
\newcommand{\fB}{\mathfrak{B}}
\newcommand{\fC}{\mathfrak{C}}
\newcommand{\fI}{\mathfrak{I}}
\newcommand{\fJ}{\mathfrak{J}}
\newcommand{\fT}{\mathfrak{T}}
\newcommand{\cA}{\mathcal{A}}
\newcommand{\cB}{\mathcal{B}}
\begin{document}
	\maketitle
	
	\begin{sol}{7.1}
		By Completeness Theorem, we can derive $\Theta \models \varphi$ from $\Theta \vdash \varphi$. Then for any $S$-interpretation $\mathfrak{I}$ 
		$$
			\fI \models \Theta \text{ implies } \fI \models \varphi
		$$
		We can construct a $S_0$-interpretation $\fI'$ by retaining the symbols occurring in $\Theta$ and $\varphi$ and keep their interpretation unchanged. By Coincidence Lemma, 
		$$
			\fI' \models \Theta \text{ implies } \fI' \models \varphi 
		$$
		It's obvious that any $S_0$-interpretation can be expanded to a $S$-interpretation without changing the interpretation of symbols in $S_0$. So for any $S_0$-interpretation $\fI'$,
		$$
					\fI' \models \Theta \text{ implies } \fI' \models \varphi 
		$$
		which means $\Theta \models \varphi$ (Now $\varphi$ is a $S_0$-formula and so does the formulas in $\Theta$ ). By Completeness Theorem, $\Theta \vdash \varphi$ and every formula occurs in the proof is a $S_0$-formula. 
		
	\end{sol}

	Let's prove the general version of Zorn' Lemma, which can be described as:
	\begin{thm}{Zorn's Lemma}
		Assume $A$ is an nonempty set and $\preceq$ is a partial order of $A$. For any chain $C \subseteq A$, there is an upper bound $s$ of $C$ such that $s \in A$. Then there exists a maximal element $c$ in $A$.
	\end{thm}
	(The Zorn's Lemma discussed in class is a special case of this theorem, where $A$ is the power set of $M$ and $\preceq$ is the $\subseteq$(subset) relation)

	\begin{sol}{7.2}\footnote{Reference: 	https://www.drmaciver.com/2015/12/direct-proofs-from-the-well-ordering-theorem}
		Let $\leq$ be a well order of $A$ and $\preceq$ be a partial order of $A$. Let's construct a function $f$
		by
		$$
		f(x) = 
		\begin{cases}
			1, & \text{for any $y\leq x, y\neq x$ and $f(y) = 1$, $x \preceq y$ }\\
			0, & \text{otherwise}
		\end{cases}
		$$
		Let 
		$$C = \{x \mid f(x) = 1\}$$
		Then $C$ is a chain for $\preceq$ because for any $x, y \in C$, $x \preceq y$ or $y \preceq x$ by the definition of $f$.
		
		By the assumption of Zorn's Lemma, there is some upper bound for $C$, call it $s$.
		
		Firstly, because $\{y \mid y\leq s,f(y)=1\} \subseteq C$, $y \preceq s$ for all $y \in C$, we must have $s \in C$.
	
		Then $s$ must be a maximal element of $A$ for relation $\preceq$. We can prove this by contradiction. If there exist $t \in A$ such that $s \neq t$ and $s \preceq t$, then $t$ must be in $C$, which contradicts that $s$ is an upper bound of $C$.
	\end{sol}

\end{document}