\documentclass[10pt,a4paper]{article}
\usepackage[utf8]{inputenc}
\usepackage{amsthm}
\usepackage{enumitem} 
\usepackage{amsmath}
\usepackage{amsfonts}
\usepackage{amssymb}
\usepackage{graphicx}

\author{Ding Yaoyao}
\title{Mathematical Logic Homework 10}

\newenvironment{sol}[1]
{\par\vspace{3mm}\noindent{\it Solution #1}.}
{\qed}

\newenvironment{thm}[1]
{\par\vspace{3mm}\noindent{\textbf{#1}}.\quad}{\\\\}

\newcommand{\abs}[1]{\lvert#1\rvert}
\newcommand{\fA}{\mathfrak{A}}
\newcommand{\fB}{\mathfrak{B}}
\newcommand{\fC}{\mathfrak{C}}
\newcommand{\fI}{\mathfrak{I}}
\newcommand{\fJ}{\mathfrak{J}}
\newcommand{\fT}{\mathfrak{T}}
\newcommand{\cA}{\mathcal{A}}
\newcommand{\cB}{\mathcal{B}}
\begin{document}
	\maketitle
	
	\begin{sol}{10.1}
		
		We call the sentences in $S_{ar}$ sentence $P_1$ to $P_7$ in order.
		
		$P_1: \forall x\neg x+1\equiv0$
		
		$P_2: \forall x\;x+0=x$
		
		$P_3: \forall x\;x\cdot0=0$
		
		$P_4: \forall x\forall y(x+1\equiv y+1 \rightarrow x\equiv y)$
		
		$P_5: \forall x\forall y\;x+(y+1) \equiv (x+y)+1$
		
		$P_6: \forall x\forall y\;x\cdot (y+1) = x\cdot y + x$
		
		$P_7: \forall x_1 \cdots\forall x_n \left( \left(\varphi\frac{0}{y}\wedge\forall y(\varphi\rightarrow\varphi\frac{y+1}{y})\right)\rightarrow \forall y \varphi \right)$
		
		By $P_2$, for all $x, y$, if $x+0 \equiv y+0$, we have $x \equiv x+0 \equiv y+0 \equiv y$, then
		\begin{equation}
			\forall x\forall y\;x+0\equiv y+0\rightarrow x\equiv y
		\end{equation}
		
		By $P_5$, $x+(z+1)\equiv y+(z+1)$ is equivalent to $(x+z)+1\equiv (y+z)+1$. By $P_4$, $x+z\equiv y+z$. Then $x+z\equiv y+z\rightarrow x\equiv y$ implies $x \equiv y$. Then
		\begin{equation}
			\forall x \forall y (x+z\equiv y+z\rightarrow x\equiv y)\rightarrow (x+(z+1)\equiv y+(z+1)\rightarrow x\equiv y)
		\end{equation}
		
		By $(1), (2), P_7$, we have
		\begin{equation}
			\forall x\forall y\forall z\;x+z\equiv y+z\rightarrow x\equiv y
		\end{equation}
		
		Let $y = 0$ in $P_5$, we have $\forall x\; x+(0+1)\equiv (x+0)+1 \equiv x + 1$. By $(3)$, we have $0+1 \equiv 1$. By $P_2$, we have
		\begin{equation}
			0+1 \equiv 1+0
		\end{equation}
		
		Because $x+1\equiv 1+x$ implies $(x+1)+1 \equiv (1+x)+1 \equiv 1+(x+1)$, we have
		\begin{equation}
			\forall x\;x+1\equiv 1+x\rightarrow (x+1)+1\equiv 1+(x+1)
		\end{equation}
		
		By $(4), (5), P_7$, we have
		\begin{equation}
			\forall x\; x+1\equiv 1+x
		\end{equation}
		
		Because $0+x\equiv x+0$ implies $0+(x+1)\equiv (0+x)+1\equiv (x+0)+1 \equiv x+1 \equiv (x+1) + 0$, we have
		\begin{equation}
			\forall x\; 0+x\equiv x+0
		\end{equation}
		
		Assume $\forall x y+x\equiv x+y$, then $(y+1)+x \equiv (1+y)+x \equiv 1+(y+x) \equiv 1+(x+y) \equiv (1+x)+y \equiv (x+1)+y \equiv x+(1+y) \equiv x+(y+1)$. Then we have
		\begin{equation}
			\forall x\; y+x\equiv x+y \rightarrow (y+1)+x\equiv x+(y+1)
		\end{equation}
		
		By $(7), (8), P_7$, we have
		$$
			\forall x \forall y \; x+y \equiv y+x
		$$
		
	\end{sol}

	
	\begin{sol}{10.2}
		Because $T$ is R-enumerable, we can construct a program $P$ enumerate $T$. Assume $\varphi_1, \varphi_2, \dots$ is an enumeration. Then we can define 
		$$
			\Theta = \{\varphi_1, \varphi_1 \wedge \varphi_2, \varphi_1\wedge\varphi_2\wedge\varphi_3, \dots\},
		$$
		which is strictly R-enumerable(Because the length of sentences increases). Then $\Theta$ is R-decidable. By the definition of modeling, we have $T \models \Theta$(thus, $\Theta \subseteq T$) and $\Theta \models T$(also $T$ is a theory, then $T = \Theta^{\models}$).Then $T$ is R-axiomatizable.
		
	\end{sol}
		
	\begin{sol}{10.3}
		\begin{align}
			\varphi_{exp}(x,y,z) = & \exists u\exists v (\varphi_{\beta}(u,v,0,1) \wedge \varphi_{\beta}(u,v,y,z) \wedge \\
			& \forall i\;(i < y \rightarrow (\forall w \varphi_{\beta}(u,v,i,w)\rightarrow\varphi_{\beta}(u,v,i+1,w\cdot x))))
		\end{align}
		
	\end{sol}


\end{document}