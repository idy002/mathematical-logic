\documentclass[10pt,a4paper]{article}
\usepackage[utf8]{inputenc}
\usepackage{amsthm}
\usepackage{amsmath}
\usepackage{amsfonts}
\usepackage{amssymb}
\usepackage{graphicx}
\author{Ding Yaoyao}
\title{Mathematical Logic Homework 1}

\newenvironment{sol}[1]
{\par\vspace{3mm}\noindent{\it Solution #1}.}
{\qed}

\newcommand{\abs}[1]{\lvert#1\rvert}

\begin{document}
	\maketitle
	
	\begin{sol}{2.1}
		$(a)\Rightarrow(b)$: If $M$ is a limited set, then there exists a bijection between $M$ and $[\abs{M}]$. If $M$ is a countable set, then there is a bijection between $M$ and $\mathbb{N}$. Because $M$ is at most countable, there exists a bijection $g$ from a subset of $\mathbb{N}$ to $M$. Then we define $f$ from $\mathbb{N}$ to $M$ such that
		\begin{equation}
		f(n) = 
		\begin{cases}
			g(n) & n \in Dom(f) \\
			\text{an arbitrary element of $M$} & n \not \in Dom(f) 
		\end{cases}
		\end{equation}
		Then $Range(f) = M$, which means $f$ is a surjection from $\mathbb{N}$ to $M$.
		
		$(b)\Rightarrow(c)$: For each $x \in M$, let $S_x = \{n \mid f(n) = x\}$. Then we define $g$ from $M$ to $\mathbb{N}$ such that
		\begin{equation}
		g(x) = \text{an arbitrary element of $S_x$}
		\end{equation}
		Because $f$ is a function from $\mathbb{N}$ to $M$, $S_x \cap S_y = \emptyset$ for different $x$ and $y$. Then $g$ is a injection from $M$ to $\mathbb{N}$.
		
		$(c)\Rightarrow(a)$: There is a bijection between $M$ and $Range(f)$. When the number of elements of $M$ is limited, $M$ is at most countable. When it's infinite, we only need to show that there is a bijection between $Range(f)$ and $\mathbb{N}$. Because $Range(f)$ is a subset of $\mathbb{N}$, we can list the elements of $Range(f)$ in a line in ascending order. Then let $g(i)$ be the $i$-th element in the line. Then $g$ is a bijection between $\mathbb{N}$ and $Range(f)$. At the same time, there is a bijection between $Range(f)$ and $M$. So there is a bijection between $\mathbb{N}$ and $M$, which means $M$ is countable. Above all, $M$ is at most countable.
	\end{sol}

	\begin{sol}{2.2}
		Firstly, it's obvious that for any $n \in \mathbb{N}$, $A^n$ is at most countable. Let $f_n$ be the bijection from $\mathbb{N}$ to $A^n$. We can place the elements in $A^*$ in such an order:
		$$
			f_0(0), f_0(1), f_1(0), f_0(2), f_1(1), f_2(0), \dots 
		$$
		(Similar method are used to prove $\mathbb{Q}$ is countable).
	\end{sol}

	\begin{sol}{2.3}
		Prove by contradiction.
		
		Assume $f$ is such a function.
		
		
		We construct a set $S$ as follows, for any $x \in M$,
		\begin{itemize}
		\end{itemize}
	\end{sol}

\end{document}